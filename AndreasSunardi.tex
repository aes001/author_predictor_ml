\documentclass[a4paper, 11pt]{article}
\usepackage{cite}
\usepackage[margin=2cm]{geometry}
\usepackage{verbatim}
\usepackage{hyperref}
\usepackage{helvet}
\renewcommand{\familydefault}{\sfdefault}

\begin{document}
\title{Categorizing the Author of Chat Messages using Machine Learning Techniques}

\author{Andreas Sunardi}

% Make title without date
\date{}
\maketitle

\section{Research hypothesis and objectives}
hwllo what do you want

Greater recall increases chance of false positive (more false identification that the text is written by J)\
Greater accuracy increases chance of false negative (more false identification that the text is not written by J)\
I don't think there is an inverse relation with these two in this case actually.

I'm using supervised learning.
multiclass classification


Set out your research idea or hypothesis. Explain why the proposed project is novel and timely, e.g. emphasizing the scientific ambition, or any potential transformative outcomes. Identify the overall aims of the project and the measurable objectives against which the outputs, outcomes and impacts of the work will be assessed. \cite{smith2021datamining}

So explain how the project is fancy and new and how it will fit within the time budget.

Explain the overall aims of the project and how you can know that you have achieved them. This is done through 

\pagebreak
\section{Background}
Introduce the background to the proposal and explain its context. Explain how this work relates to past research and current projects at Leeds University, in the UK and abroad. Identify the key research questions that the project will address.\cite{smith2021dataminingOther}

\pagebreak
\section{Importance and contribution to knowledge}
Explain how the project may contribute to current or
future economic success; to future development of key emerging industries; or addresses key
societal challenges. Describe how your research would benefit national and international research,
and engage with research in other disciplines to broaden the reach of the new knowledge.

\pagebreak
\section{Pilot study}
Describe your implementation of an initial pilot study to demonstrate feasibility of the project, including selection of a simple case study, acquiring sample data, development and evaluation of your prototype solution.

\pagebreak
\section{Programme and methodology}
Describe the work programme including research and user
evaluation. Identify the contribution of each member of the research team including users and/or stakeholders. Provide milestones and deliverables that you will use to monitor progress, and explain how the project will be managed.

The research work programme should make use of an appropriate methodology for AI projects, such as CRISP-DM; and should include use of at least two data mining and/or text analytics methods, tools or techniques introduced in the module (eg SketchEngine, Weka, ChatGPT)

\pagebreak
\section{Workplan diagram}
Must match the written description of the research work programme, showing start, end and duration of each phase or work-package.

\pagebreak
\appendix
\section{Appendix}
Describe your use of data mining and text analytics tools in developing your Report. This could include: tools used in the small pilot study to trial the methods proposed; use of tools like Google Scholar or ChatGPT in searching for background information and drafting the report (include examples of query and results); use of tools like Grammarly, Word or ChatGPT to check and correct grammar and style.


\pagebreak
% We are going to use bibtex to manage the references from a refx.bib file
\bibliographystyle{plain}
\bibliography{refx}

\end{document}